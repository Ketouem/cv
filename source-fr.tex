% Good tuto for moderncv available here: http://blog.madrzejewski.com/creer-cv-elegant-latex-moderncv/

% Calling the package moderncv
\documentclass[10pt,a4paper,sans,colorlinks,linkcolor=true]{moderncv}

\usepackage[T1]{fontenc}
\usepackage[default]{raleway}

% modercv configuration
\moderncvtheme[blue]{classic}

% Set date column width
\setlength{\hintscolumnwidth}{3cm}

% Accented characters
\usepackage[utf8]{inputenc}

% Use of package geometry that allows for margin control
\usepackage[top=0.9cm, bottom=0.9cm, left=1.0cm, right=1.0cm]{geometry}

% Image inclusion
\usepackage{graphicx}

\usepackage{tikz}
\usetikzlibrary{tikzmark}


% Column width (for dates)
\setlength{\hintscolumnwidth}{3cm}

% Headers part
\firstname{\textcolor{black}{Cyril}}
\familyname{\textcolor{black}{Thomas}}
\address{\textcolor{black}{Boulogne-Billancourt - France}}
\email{ketouem@gmail.com}
\homepage{ketouem.com}
\extrainfo{\textcolor{black}{33 ans - Titulaire du permis B} \\ \\
           \href{https://goo.gl/0q4mqd}{\includegraphics[scale=0.50]{./images/linkedin.png}}
           \href{https://goo.gl/HmYaEp}{\includegraphics[scale=0.50]{./images/github.png}}
           \href{https://goo.gl/5I6niN}{\includegraphics[scale=0.50]{./images/stackoverflow.png}}
           \href{https://goo.gl/HW9dKv}{\includegraphics[scale=0.50]{./images/codewars.png}}\tikzmark{end}}

\begin{document}
\hypersetup{
  colorlinks=true,
  urlcolor=cyan,
  pdftitle={Cyril THOMAS - CV},
  pdfauthor={Cyril THOMAS}
}

% Putting a backcolor in the header
\begin{tikzpicture}[remember picture,overlay]
\fill[blue!10]
  (current page.north west) rectangle ([yshift=-1cm]current page.east|-{pic cs:end});
\end{tikzpicture}

% Include an image inside the section, redefining section in order to include an additional parameter
\makeatletter
\RenewDocumentCommand{\section}{smo}{%
  \par\addvspace{2.5ex}%
  \phantomsection{}% reset the anchor for hyperrefs
  \addcontentsline{toc}{section}{#2}%
  \parbox[t]{\hintscolumnwidth}{\strut\raggedleft\raisebox{\baseletterheight}  {\color{color1}\rule{\hintscolumnwidth}{0.95ex}}}%
  \hspace{\separatorcolumnwidth}%
    \IfNoValueTF{#3}
    {\parbox[t]{\maincolumnwidth}{\strut\sectionstyle{#2}}}
    {\parbox[t]{\maincolumnwidth}{\strut#3\ \sectionstyle{#2}}}%
  \par\nobreak\addvspace{1ex}\@afterheading%
}
\makeatother

%
\makecvtitle

\bigbreak

% Professional stuff
\section{Expériences Professionnelles}[\includegraphics[scale=0.40]{./images/worker.png}]
% Dalma
\cventry{Depuis Novembre 2023}{Senior Solutions Architect}{Dalma}{Paris}{France}{%
}
% Viceversa
\cventry{Mars 2022 - Octobre 2023}{Senior Cloud Engineer / DevOps}{Viceversa (ex Barney)}{Paris}{France}{%
\begin{itemize}
    \item Accompagnement de l'équipe de développement dans le choix et l'utilisation des ressources \& services \textbf{AWS}
    \item Restructuration du setup \textbf{AWS} en organisation multi-comptes avec gestion des droits d'accès selon les profils
    \item Sécurisation des ressources exposées (mise en place de \textbf{Web ACLs WAF}, enclavement réseau d'instance, configuration des \textbf{VPCs})
    \item Monitoring de la workload \textbf{Lambda} (métriques \textbf{CloudWatch} \& analyse des logs) et des instances \textbf{EC2}
    \item Mise en place d'un projet IaC \textbf{terraform/terragrunt} afin de formaliser l'architecture cloud
    \item Création d'un projet \textbf{CDK} (TypeScript) pour permettre un déploiement facilité des stacks de développement
    \item Création de \textbf{GitHub actions} permettant de faciliter le workflow de l'équipe de développement
\end{itemize}
}
\bigbreak
% Enovacom
\cventry{Juillet 2017 - Décembre 2021}{Senior Software Engineer / DevOps}{Enovacom}{La Défense}{France}{%
\begin{itemize}
    \item Ré-architecture de la solution existante sous \textbf{AWS} avec prise en compte des contraintes de coûts/sécurité/performance/maintenabilité
    \item Participation à l'élaboration d'une architecture basée sur \textbf{serverless} (API Gateway, Lambda, etc.)
    \item Gestion des incidents et maintien en condition opérationnelle de la charge de production
    \item Développement de modules \textbf{terraform/terragrunt} pour la description de l'infrastructure
    \item Création d'un pipeline permettant la génération automatique d'une documentation utilisateur et leur publication sur un site dédié hébergé via AWS S3
    \item Maintien et amélioration de playbooks Ansible permettant le provisioning et la configuration applicative pour des machines virtuelles EC2 / VMWare (esx)
\end{itemize}
}
\bigbreak
% WayzUp
\cventry{Octobre 2016 - Mai 2017}{Software Developer}{WayzUp}{Paris}{France}{%
\begin{itemize}
    \item Création d'un dashboard \textbf{Sinatra}, hébergé sur \textbf{Heroku}, affichant le résultat des builds \textbf{CircleCI}
    \item Mise en place d'un outil basé sur \textbf{docker} (\textbf{machine} \& \textbf{compose}) permettant le montage et stockage sur \textbf{S3} de dumps de bases de données de production sur un environnement de développement
    \item Configuration de distributions \textbf{Cloudfront} pour la distribution d'assets
    \item Gestion et suivi des incidents de production via \textbf{Bugsnag}
    \item Développement d'une couche CRM backoffice et intégration à l'interface \textbf{ActiveAdmin} existante
    \item Correction des failles de sécurité à l'aide de \textbf{Gemnasium}
    \item Maintien et évolutions fullstack des applicatifs (\textbf{Sinatra}, \textbf{Rails})
\end{itemize}
}
\bigbreak
% Tinyclues
\cventry{Mars 2016 - Septembre 2016}{Software Engineer}{Tinyclues}{Paris}{France}{%
\begin{itemize}
    \item Création d'un webservice \textbf{Flask} permettant le tracking d'évènements (clicks, page views, etc.)
    \item Participation à l'évolution d'une stack de data science
    \item Mise en place d'environnements de développement utilisant \textbf{Docker} \& \textbf{docker-compose}
    \item Améliorations et optimisations du traitement de flux de données entrant via \textbf{PyTables}, \textbf{HDF} \& \textbf{pandas}
\end{itemize}
}
\bigbreak
% Optiflows
\cventry{Septembre 2014 - Février 2016}{Software Developer \& Architect}{Optiflows}{Boulogne-Billancourt}{France}{%
\begin{itemize}
    \item Création d'applications et de web services sous \textbf{Flask} \& \textbf{Django}
    \item Mise en place d'environnements de développement utilisant \textbf{Docker} \& \textbf{docker-compose}
    \item Maintien et évolution d'applications \textbf{Python} \& \textbf{Go}
    \item Participation à l'élaboration d'une architecture multi-agents basée sur \textbf{Docker} permettant le rappel de rendez-vous patients via mail \& sms
    \item Mise en place d'un pipeline de build via \textbf{Jenkins} \& \textbf{CircleCI}
\end{itemize}
}
\bigbreak
% SmartFocus
\cventry{Novembre 2011 - Juin 2014}{Analyste Qualité Logicielle}{SmartFocus}{Clichy la Garenne}{France}{%
\begin{itemize}
	\item Développement d'un framework \textbf{Python} pour l'automatisation des tests fonctionnels basé sur \textbf{Selenium Webdriver}
	\item Développement d'outils internes permettant de faciliter les tests (\textbf{Python}/\textbf{Flask}/\textbf{SQLAlchemy} \& Java/GWT)
	\item Intégration des outils de tests automatiques dans une chaîne de build via \textbf{Jenkins}
	\item Utilisation (\textbf{Gatling}, \textbf{locust}, \textbf{loadUI}) et développement d'outils permettant les tests de charge
	\item Rédaction et exécution des plans de tests (fonctionnels et de charge)
	\item Création et suivi des rapports d'incidents sur \textbf{Jira}
\end{itemize}
}
\bigbreak
% Groupe Mutualiste RATP
\cventry{Juin - Août 2009 \& 2010}{Développeur (Stagiaire)}{Groupe Mutualiste RATP}{Paris}{France}{%
\begin{itemize}
	\item Développement d'outils de backoffice (gestion de stock, mailing clients) en \textbf{Delphi} (Object Pascal) avec un stockage des données dans une base \textbf{PostgreSQL}
\end{itemize}
}

\bigbreak

% Education stuff
\section{Diplômes et Études}[\includegraphics[scale=0.50]{./images/graduation-hat-and-diploma.png}]
\cventry{2010-2011}{MSc in Computing Science - HETAC (H1)}{Griffith College}{Dublin}{Irlande}{%
\begin{itemize}
	\item Mémoire: "\href{http://goo.gl/iyzQsX}{\textit{Relevance of glyph recognition and augmented reality in desktop applications with AForge.NET and GRATF.}}"
\end{itemize}
}
\bigbreak
\cventry{2008-2011}{Expert en Informatique et Système d'Information (Niveau I - Ingénieur)}{EPSI Paris}{Levallois-Perret}{France}{}

\bigbreak

% Certifications
\section{Certifications}[\includegraphics[scale=0.50]{./images/document.png}]
\cventry{2019}{\href{https://public-dist.s3.amazonaws.com/aws-certified-solutions-architect-associate-certificate.pdf}{Certified Solutions Architect Associate}}{Amazon Web Services}{}{}{}
\cventry{2021}{\href{https://public-dist.s3.amazonaws.com/aws-certified-developer-associate-certificate.pdf}{Certified Developer Associate}}{Amazon Web Services}{}{}{}

\bigbreak

% Skills - Ops
\begin{minipage}[t]{0.5\textwidth}
\section{Ops}[\includegraphics[scale=0.40]{./images/wrench.png}]
\cvitem{Cloud}{AWS}
\cvitem{Configuration}{Ansible, Terraform, Terragrunt, Packer}
\cvitem{Framework}{Serverless}
\cvitem{CI}{CircleCI, Jenkins}
\cvitem{PaaS}{Heroku}
\cvitem{Container}{Docker (compose \& machine)}
\cvitem{Sécurité}{Gemnasium}
\cvitem{Gestion incidents}{Bugsnag, Sentry}
\end{minipage}
% Skills - Python
\begin{minipage}[t]{0.5\textwidth}
\section{Python}[\includegraphics[scale=0.40]{./images/python.png}]
\cvitem{Web}{Flask, aiohttp, Django}
\cvitem{Persistance}{SQLAlchemy, cx\_Oracle, PyMongo}
\cvitem{Test}{Nose, Mock, Locust}
\cvitem{Job processing}{Celery, asyncio}
\cvitem{Comp. Vision}{OpenCV, PIL}
\end{minipage}

\bigbreak

% Skills - Ruby
\begin{minipage}[t]{0.5\textwidth}
\section{Ruby}[\includegraphics[scale=0.40]{./images/ruby.png}]
\cvitem{Web}{Rails, Sinatra}
\cvitem{Persistance}{ActiveRecord}
\cvitem{Test}{rspec}
\cvitem{Job processing}{Sidekiq}
\cvitem{Backoffice}{ActiveAdmin}
\end{minipage}
% Skills -
\begin{minipage}[t]{0.5\textwidth}
\section{Compétences Générales}[\includegraphics[scale=0.40]{./images/code.png}]
\cvitem{Front-end}{jQuery, Bootstrap}
\cvitem{SCM}{Git, SVN}
\cvitem{SGBD}{PostgreSQL, MongoDB}
\cvitem{OS}{Linux (Debian/Ubuntu), macOS}
\cvitem{Gestion}{Jira, TestLink}
\cvitem{Fonctionnel}{Selenium, soapUI, TestComplete}
\cvitem{Charge}{locust, Gatling}
\end{minipage}

\bigbreak

% Languages
\begin{minipage}[t]{0.5\textwidth}
\section{Langues}[\includegraphics[scale=0.40]{./images/talk.png}]
\cvlanguage{Anglais}{lu, écrit, parlé}{}
\cvlanguage{Français}{langue maternelle}{}
\end{minipage}
% Hobbies
\begin{minipage}[t]{0.5\textwidth}
\section{Hobbies}[\includegraphics[scale=0.40]{./images/running-man.png}]
\cvitem{Course à pied}{10 kms \& semi-marathon (\href{https://public-dist.s3.amazonaws.com/semi_2018.pdf}{1h24}) }
\end{minipage}

\end{document}
